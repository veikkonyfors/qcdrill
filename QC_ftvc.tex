\documentclass[11pt, fleqn]{article}
\usepackage{amsmath}
\usepackage{graphicx}
\usepackage{hyperref}
\usepackage[latin1]{inputenc}
\usepackage[ddmmyyyy]{datetime}
\usepackage[braket, qm]{qcircuit}
\renewcommand{\dateseparator}{.}


\title{Quantum Computing for the very curious, summary}
\author{Veikko Nyfors}
\date{\today}

\begin{document}
\maketitle

\paragraph{In this document I was originally to wrap up the Strangeworks' paced repetition document named 'Quantum computing for the very curious'.
Eventually ended up on wrapping up also information available in Qiskit's and Azure's documentation.
Items incorporated are ones that I felt I was going to need as available supplementary material along the road.
Exspecially on the first few hundred meters, where composing this document was a very good practical excersize to learn the stuff too.
Kind of killing two birds with one stone.
\\
Hopefully this document is helpfull fo other people as well. Third bird? :-)
}


\newpage

\section{Beneficial matrix fundamentals}

\bf{Dirac notation}
\[|0\rangle=\begin{bmatrix} 1\\0\end{bmatrix}=e_0,\:
|1\rangle=\begin{bmatrix} 0\\1\end{bmatrix}=e_1,\:
\langle0|=\begin{bmatrix} 1&0\end{bmatrix},\:\langle1|=\begin{bmatrix} 0&1\end{bmatrix}\]
\[|\psi\rangle^\dagger=\langle\psi|\]
\\~\\
\bf{Matrix addition}
\[\frac{1+i}{2}|0\rangle + \frac{i}{\sqrt{2}}|1\rangle = \begin{bmatrix} \frac{1+i}{2}\\\frac{i}{\sqrt{2}} \end{bmatrix}\]
\\~\\
\bf{Unitarity}
\[U^\dagger U=I,\quad U^\dagger=(U^T)^*,\quad
\begin{bmatrix}a & b\\c & d\end{bmatrix}^\dagger=\begin{bmatrix}a^* & c^*\\d^* & d^*\end{bmatrix},\:I=\begin{bmatrix}1&0\\0&1\end{bmatrix}\]
\\~\\
\bf{Tensor product\\}

$|00\rangle=|0\rangle \otimes |0\rangle=
\begin{bmatrix}1\\0 \end{bmatrix}\otimes
\begin{bmatrix}1\\0 \end{bmatrix}=
\begin{bmatrix}1\begin{bmatrix}1\\0 \end{bmatrix}\\0\begin{bmatrix}1\\0 \end{bmatrix}\end{bmatrix}=
\begin{bmatrix}1\\0\\0\\0 \end{bmatrix}$
\\~\\

$|01\rangle=|0\rangle \otimes |1\rangle=
\begin{bmatrix}1\\0 \end{bmatrix}\otimes
\begin{bmatrix}0\\1 \end{bmatrix}=
\begin{bmatrix}1\begin{bmatrix}0\\1 \end{bmatrix}\\0\begin{bmatrix}0\\1 \end{bmatrix}\end{bmatrix}=
\begin{bmatrix}0\\1\\0\\0 \end{bmatrix}$
\\~\\

$|10\rangle=|1\rangle \otimes |0\rangle=
\begin{bmatrix}0\\1 \end{bmatrix}\otimes\begin{bmatrix}1\\0 \end{bmatrix}=
\begin{bmatrix}0\begin{bmatrix}1\\0 \end{bmatrix}\\1\begin{bmatrix}1\\0 \end{bmatrix}\end{bmatrix}=
\begin{bmatrix}0\\0\\1\\0 \end{bmatrix}$
\\~\\

$|11\rangle=|1\rangle \otimes |1\rangle=
\begin{bmatrix}0\\1 \end{bmatrix}\otimes\begin{bmatrix}0\\1 \end{bmatrix}=
\begin{bmatrix}0\begin{bmatrix}0\\1 \end{bmatrix}\\1\begin{bmatrix}0\\1 \end{bmatrix}\end{bmatrix}=
\begin{bmatrix}0\\0\\0\\1 \end{bmatrix}$
\\~\\

$\sqrt{2}*H\otimes \sqrt{2}*H = \begin{bmatrix}1&1\\1&-1\end{bmatrix}\otimes
\begin{bmatrix}1&1\\1&-1\end{bmatrix} =
\begin{bmatrix}1\begin{bmatrix}1&1\\1&-1\end{bmatrix}&1\begin{bmatrix}1&1\\1&-1\end{bmatrix}\\1\begin{bmatrix}1&1\\1&-1\end{bmatrix}&-1\begin{bmatrix}1&1\\1&-1\end{bmatrix}\end{bmatrix}=
\begin{bmatrix}1&1&1&1\\1&-1&1&-1\\1&1&-1&-1\\1&-1&-1&1\\\end{bmatrix}
\\~\\$


\bf{Density operator\\}

$
for \hspace{0.5em} |\psi \rangle = \begin{bmatrix} a_0 \\ a_1 \\a_3 \\ a_4 \end{bmatrix}, \hspace{0.5em}
Density \hspace{0.5em} operator \hspace{0.5em} \rho \equiv  | \psi \rangle \langle \psi | \hspace{0.5em} (\sim Outer \hspace{0.5em} product)=\\
\begin{bmatrix} a_0 \\ a_1 \\a_2 \\ a_3 \end{bmatrix} \begin{bmatrix} a_0^* & a_1^* & a_2^* & a_3^* \end{bmatrix} =
\begin{bmatrix} |a_0|^2 & a_0  a_1^*& a_0  a_2^*  & a_0  a_3^*  \\ a_1  a_0^*& |a_1|^2 & a_1  a_2^*  & a_1  a_3^* \\ 
a_2  a_0^*& a_2  a_1^*  & |a_2|^2 & a_2  a_3^* \\ a_3  a_0^*& a_3  a_1^*& a_3  a_2^*  & |a_3|^2 \end{bmatrix}
$


\section{Gates}
\subsection{Pauli gates: X(/NOT), Y and Z}
\[ NOT|0\rangle = |1\rangle \quad NOT|1\rangle = |0\rangle,\quad NOT (\alpha|0\rangle+\beta|1\rangle)=\alpha|0\rangle+\beta|1\rangle,\quad\]
\[X =\begin{bmatrix}0 & 1\\1 & 0\end{bmatrix},\:
\begin{bmatrix}0 & 1\\1 & 0\end{bmatrix}\begin{bmatrix} 1&0\end{bmatrix}=\begin{bmatrix} 0&1\end{bmatrix},\:
\begin{bmatrix}0 & 1\\1 & 0\end{bmatrix}\begin{bmatrix} 0&1\end{bmatrix}=\begin{bmatrix} 1&0\end{bmatrix}\]
\[XX|\psi\rangle=\psi\]
\[Y=\begin{bmatrix}0&-i\\i&0\end{bmatrix},\:Y|0\rangle = i|1\rangle,\:Y|1\rangle =-i|0\rangle\]
\[Z=\begin{bmatrix}1&0\\0&1\end{bmatrix},Z|0\rangle = |0\rangle,Z|1\rangle =-|1\rangle\]
\\~\\
\subsection{Hadamard Gate}
\[H|0\rangle = \frac{|0\rangle +|1\rangle}{\sqrt 2}=\begin{bmatrix}\frac{1}{\sqrt 2}\\0\end{bmatrix}+
\begin{bmatrix}0\\\frac{1}{\sqrt 2}\end{bmatrix}=
\begin{bmatrix}\frac{1}{\sqrt 2}\\\frac{1}{\sqrt 2}\end{bmatrix}\]
\[H|1\rangle = \frac{|0\rangle -|1\rangle}{\sqrt 2}=
\begin{bmatrix}\frac{1}{\sqrt 2}\\0\end{bmatrix}-
\begin{bmatrix}0\\\frac{1}{\sqrt 2}\end{bmatrix}=
\begin{bmatrix}\frac{1}{\sqrt 2}\\-\frac{1}{\sqrt 2}\end{bmatrix}
\]


\[H(\alpha|0\rangle+\beta|1\rangle)=\alpha\left(\frac{|0\rangle+|1\rangle}{\sqrt 2}\right)+\beta\left(\frac{|0\rangle-|1\rangle}{\sqrt 2}\right)
=\frac{\alpha+\beta}{\sqrt 2}|0\rangle+\frac{\alpha-\beta}{\sqrt 2}|1\rangle\]
\[H=\frac{1}{\sqrt 2}\begin{bmatrix}1&1\\1&-1\end{bmatrix},\:H=H^\dagger,\:HH=HH^\dagger=I\]
\[\Rightarrow HH|0\rangle=|0\rangle,\:HH|1\rangle=|1\rangle,\:HH|\psi\rangle=|\psi\rangle\] 
\vspace{0.5em}

Applying Hadamard to a qubit in Hadamard state, gives back the basis state Hadamard was originally applied to
\\~\\
\Qcircuit @C=1em @R=2em {&& \frac{\ket{0} + \ket{1}}{\sqrt 2} &&& \gate{H} & \measureD{\mbox{M}} & = & \ket{0}} \vspace{1em}
\Qcircuit @C=1em @R=2em {&& \frac{\ket{0} - \ket{1}}{\sqrt 2} &&& \gate{H} & \measureD{\mbox{M}} & = & \ket{1}}
\vspace{1em}


\subsection{CNOT Gate}
\vspace{1em}
\begin{Large}

\Qcircuit @C=1em @R=1em {
&& \lstick{\ket{1}} & \ctrl{1} & \rstick{\ket{1}} \qw &&&&& \lstick{\ket{1}} & \ctrl{1} & \rstick{\ket{1}} \qw\\
&& \lstick{\ket{1}} & \targ & \rstick{\ket{0}} \qw &&&&& \lstick{\ket{0}} & \targ & \rstick{\ket{1}} \qw
\\~\\
}
\vspace{1em}

\Qcircuit @C=1em @R=1em {
&& \lstick{\ket{0}} & \ctrl{1} & \rstick{\ket{0}} \qw &&&&& \lstick{\ket{0}} & \ctrl{1} & \rstick{\ket{0}} \qw\\
&& \lstick{\ket{1}} & \targ & \rstick{\ket{1}} \qw &&&&& \lstick{\ket{0}} & \targ & \rstick{\ket{0}} \qw
\\~\\
}

$
CNOT=\begin{bmatrix}1&0&0&0\\0&1&0&0\\0&0&0&1\\0&0&1&0\\\end{bmatrix}
\\~\\$

\bf{In general superposition, doesn't seem to modify control bit\\}
\Qcircuit @C=1em @R=1em {&&&&&&&&&&&&& \alpha\ket{00}+\beta\ket{01}+\gamma\ket{10}+\delta\ket{11}
\rightarrow \alpha\ket{00}+\beta\ket{01}+\gamma\ket{11}+\delta\ket{10}}
\vspace{1em}

\bf{But in this specific Hadamard case, seemingly does it\\}\\~\\
\bf{A sample of CNOT changing the control qubit\\}\\~\\
\Qcircuit @C=1em @R=1em {
&&& \lstick{\frac{\ket{0} + \ket{1}}{\sqrt 2}} & \ctrl{1} & \qw & \rstick{\frac{\ket{0} - \ket{1}}{\sqrt 2}} \qw \\
&&& \lstick{\frac{\ket{0} - \ket{1}}{\sqrt 2}} & \targ & \qw  & \rstick{\frac{\ket{0} - \ket{1}}{\sqrt 2}} \qw
}
\\~\\or\\\
\Qcircuit @C=1em @R=1em {
& \ket{+} && \ctrl{1} & \qw && \ket{-} \\
& \ket{-} && \targ{1} & \qw && \ket{-}
}
\vspace{1em}

\\~\\ Proof
$\\~\\\begin{bmatrix}\frac{1}{\sqrt 2}\\\frac{1}{\sqrt 2}\end{bmatrix}\otimes
\begin{bmatrix}\frac{1}{\sqrt 2}\\-\frac{1}{\sqrt 2}\end{bmatrix}=
\begin{bmatrix}\frac{1}{\sqrt 2}\begin{bmatrix}\frac{1}{\sqrt 2}\\-\frac{1}{\sqrt 2} \end{bmatrix}
\\\frac{1}{\sqrt 2}\begin{bmatrix}\frac{1}{\sqrt 2}\\-\frac{1}{\sqrt 2} \end{bmatrix}\end{bmatrix}=
\begin{bmatrix}1/2\\-1/2\\1/2\\-1/2 \end{bmatrix}
$

\\~\\
Apply CNOT on this\\~\\
$\begin{bmatrix}1&0&0&0\\0&1&0&0\\0&0&0&1\\0&0&1&0\\\end{bmatrix}
\begin{bmatrix}1/2\\-1/2\\1/2\\-1/2 \end{bmatrix}=
\begin{bmatrix}1/2\\-1/2\\-1/2\\1/2 \end{bmatrix}=
\begin{bmatrix}\frac{1}{\sqrt 2}\begin{bmatrix}\frac{1}{\sqrt 2}\\-\frac{1}{\sqrt 2} \end{bmatrix}
\\-\frac{1}{\sqrt 2}\begin{bmatrix}\frac{1}{\sqrt 2}\\-\frac{1}{\sqrt 2} \end{bmatrix}\end{bmatrix}=\\~\\
\begin{bmatrix}\frac{1}{\sqrt 2}\\-\frac{1}{\sqrt 2}\end{bmatrix}\otimes
\begin{bmatrix}\frac{1}{\sqrt 2}\\-\frac{1}{\sqrt 2}\end{bmatrix}=
|-\rangle \otimes |-\rangle=
|--\rangle$

\\~\\\\~\\
Another drill
\\~\\\\~\\
\Qcircuit @C=1em @R=1em {
& \ket{0} && \gate{H} \qw & \ctrl{1} & \qw & \gate{H} & \qw \\
& \ket{0} && \qw & \targ{1} & \qw & \qw & \qw
}
\\~\\\\~\\

$|\psi\rangle_{init} = |0\rangle \otimes |0\rangle = |00\rangle\\~\\$

Apply Hadamard on first qubit\\~\\
$|\psi\rangle_{0} = \frac{|0\rangle +|1\rangle}{\sqrt 2}\otimes |0\rangle=
\frac{1}{\sqrt 2}\left(\begin{bmatrix}1\\0 \end{bmatrix}+\begin{bmatrix}0\\1 \end{bmatrix}\right)\otimes |0\rangle=\\~\\
\frac{1}{\sqrt 2}\left(\begin{bmatrix}1\\1 \end{bmatrix}\right)\otimes|0\rangle=
\begin{bmatrix}\frac{1}{\sqrt 2}\\\frac{1}{\sqrt 2} \end{bmatrix}\otimes \begin{bmatrix}1\\0 \end{bmatrix}=
\begin{bmatrix}\frac{1}{\sqrt 2}\begin{bmatrix}1\\0 \end{bmatrix}\\\frac{1}{\sqrt 2}\begin{bmatrix}1\\0 \end{bmatrix}\end{bmatrix}=
\frac{1}{\sqrt 2}\begin{bmatrix}1\\0\\1\\0 \end{bmatrix}=
\frac{1}{\sqrt 2}\left(\begin{bmatrix}1\\0\\0\\0 \end{bmatrix}+
\begin{bmatrix}0\\0\\1\\0 \end{bmatrix}\right)=
\frac{1}{\sqrt 2}\left(\begin{bmatrix}1\\0\\0\\0 \end{bmatrix}+
\begin{bmatrix}0\\0\\1\\0 \end{bmatrix}\right)=
\frac{1}{\sqrt 2}\left(|00\rangle+|10\rangle\right)=
\frac{|00\rangle+|10\rangle}{\sqrt 2}\\~\\
$

Apply CNOT on the above result\\~\\
$\begin{bmatrix}1&0&0&0\\0&1&0&0\\0&0&0&1\\0&0&1&0\\\end{bmatrix} \frac{1}{\sqrt 2}\begin{bmatrix}1\\0\\1\\0 \end{bmatrix} =
\frac{1}{\sqrt 2}\begin{bmatrix}1\\0\\0\\1 \end{bmatrix}=
\frac{1}{\sqrt 2}\left(\begin{bmatrix}1\\0\\0\\0 \end{bmatrix}+
\begin{bmatrix}0\\0\\0\\1 \end{bmatrix}\right)=
\frac{1}{\sqrt 2}\left(|00\rangle+|11\rangle\right)$

\end{Large}

\vspace{1em}
\subsection{Rotation}
\[R_y(\theta) =\begin{bmatrix}\cos(\theta/2) & -\sin(\theta/2)\\ \sin(\theta/2) & \cos(\theta/2)\end{bmatrix}\]

\newpage
\section{Multiple qubits}

Two-qubit state is given by the tensor (or Kronecker) product

$|0\rangle \otimes |1\rangle=
\begin{bmatrix}1\\0 \end{bmatrix}\otimes
\begin{bmatrix}0\\1 \end{bmatrix}=
\begin{bmatrix}1\begin{bmatrix}0\\1 \end{bmatrix}\\0\begin{bmatrix}0\\1 \end{bmatrix}\end{bmatrix}=
\begin{bmatrix}0\\1\\0\\0 \end{bmatrix}
\\~\\$

$H\otimes H = 
\frac{1}{\sqrt 2}\begin{bmatrix}1&1\\1&-1\end{bmatrix}\otimes
\frac{1}{\sqrt 2}\begin{bmatrix}1&1\\1&-1\end{bmatrix} =
\begin{bmatrix}\frac{1}{\sqrt 2}&\frac{1}{\sqrt 2}\\\frac{1}{\sqrt 2}&-\frac{1}{\sqrt 2}\end{bmatrix}\otimes
\begin{bmatrix}\frac{1}{\sqrt 2}&\frac{1}{\sqrt 2}\\\frac{1}{\sqrt 2}&-\frac{1}{\sqrt 2}\end{bmatrix}=
\\~\\
\begin{bmatrix}\frac{1}{\sqrt 2}\begin{bmatrix}\frac{1}{\sqrt 2}&\frac{1}{\sqrt 2}\\\frac{1}{\sqrt 2}&-\frac{1}{\sqrt 2}\end{bmatrix}&\frac{1}{\sqrt 2}\begin{bmatrix}\frac{1}{\sqrt 2}&\frac{1}{\sqrt 2}\\\frac{1}{\sqrt 2}&-\frac{1}{\sqrt 2}\end{bmatrix}\\\frac{1}{\sqrt 2}\begin{bmatrix}\frac{1}{\sqrt 2}&\frac{1}{\sqrt 2}\\\frac{1}{\sqrt 2}&-\frac{1}{\sqrt 2}\end{bmatrix}&-\frac{1}{\sqrt 2}\begin{bmatrix}\frac{1}{\sqrt 2}&\frac{1}{\sqrt 2}\\\frac{1}{\sqrt 2}&-\frac{1}{\sqrt 2}\end{bmatrix}\end{bmatrix}=
\begin{bmatrix}
\frac{1}{2}&\frac{1}{2}&\frac{1}{2}&\frac{1}{2}\\
\frac{1}{2}&-\frac{1}{2}&\frac{1}{2}&-\frac{1}{2}\\
\frac{1}{2}&\frac{1}{2}&-\frac{1}{2}&-\frac{1}{2}\\
\frac{1}{2}&-\frac{1}{2}&-\frac{1}{2}&\frac{1}{2}\\
\end{bmatrix}
\\~\\$

E.g. for below two qubit Hadamard system

$|H|0\rangle |H|1\rangle\rangle =
H|0\rangle \otimes H|1\rangle =
\begin{bmatrix}\frac{1}{\sqrt 2}\\\frac{1}{\sqrt 2}\end{bmatrix} \otimes
\begin{bmatrix}\frac{1}{\sqrt 2}\\-\frac{1}{\sqrt 2}\end{bmatrix} =
\begin{bmatrix}\frac{1}{\sqrt 2}*\frac{1}{\sqrt 2}\\
\frac{1}{\sqrt 2}*-\frac{1}{\sqrt 2}\\
\frac{1}{\sqrt 2}*\frac{1}{\sqrt 2}\\
\frac{1}{\sqrt 2}*-\frac{1}{\sqrt 2}\end{bmatrix}=
\begin{bmatrix}\frac{1}{2}\\
-\frac{1}{2}\\
\frac{1}{2}\\
-\frac{1}{2}\end{bmatrix}
\\~\\$

In other words

$H\otimes H(|01\rangle) =
H^{\otimes 2} \left( \begin{bmatrix}1 \\ 0 \end{bmatrix}\otimes \begin{bmatrix}0\\1 \end{bmatrix} \right) =
\frac{1}{2}\begin{bmatrix}1 &1 &1 &1 \\ 1 &-1 &1 &-1 \\ 1 &1 &-1 &-1 \\ 1 &-1 &-1 &1 \end{bmatrix}\begin{bmatrix}0\\ 1\\ 0\\ 0\end{bmatrix} =
\frac{1}{2}\begin{bmatrix}1\\ -1\\ 1\\ -1\end{bmatrix}$

\pagebreak
\section{Entanglement}

\bf{A pure two qubit system's state is expressed as tensor product of each qubit's state vector. If we have a two qubit system, whose state is not possible to express
as a tensor product of two such subvectors, the state is entangled. I.e. state has to be factorizable (or separable) in tensorproduct sense for the state to be unentangled.
\\~\\}
\Qcircuit @C=1em @R=1em {&&&&&& a\ket{00}+b\ket{01}+c\ket{10}+d\ket{11}}

\vspace{1em}

\bf{If above general superposition is separable, then it should be possible to represent it as\\~\\}
$(\alpha\ket{0}+\beta\ket{1})\otimes(\gamma\ket{0}+\delta\ket{1})=
\left(\begin{bmatrix}\alpha \\ 0 \end{bmatrix}+\begin{bmatrix}0 \\ \beta \end{bmatrix}\right)\otimes\left(\begin{bmatrix}\gamma \\ 0 \end{bmatrix}+\begin{bmatrix}0 \\ \delta \end{bmatrix}\right)=\\~\\
\begin{bmatrix}\alpha\\\beta \end{bmatrix}\otimes\begin{bmatrix}\gamma\\\delta \end{bmatrix}=
\begin{bmatrix}\alpha\begin{bmatrix}\gamma\\\delta \end{bmatrix}\\\beta\begin{bmatrix}\gamma\\\delta \end{bmatrix}\end{bmatrix}=
\begin{bmatrix}\alpha\gamma\\\alpha\delta\\\beta\gamma\\\beta\delta \end{bmatrix}=
(\alpha\gamma|00\rangle+\alpha\delta|01\rangle+\beta\gamma|10\rangle+\beta\delta|11\rangle)
\\~\\$

This is the same as our initial pure state if and only if
$\\~\\\alpha\gamma=a\land\alpha\delta=b\land\beta\gamma=c\land\beta\delta=d \Leftrightarrow ad=bc \Leftrightarrow ad-bc=0
$


\end{document}
