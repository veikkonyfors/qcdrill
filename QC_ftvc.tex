\documentclass[11pt, fleqn]{article}
\usepackage{amsmath}
\usepackage{graphicx}
\usepackage{hyperref}
\usepackage[latin1]{inputenc}
\usepackage[ddmmyyyy]{datetime}
\usepackage[braket, qm]{qcircuit}
\renewcommand{\dateseparator}{.}


\title{Quantum Computing for the very curious, summary}
\author{Veikko Nyfors}
\date{\today}

\begin{document}
\maketitle

In this document I am wrapping up the Strangeworks' paced repetition document named 'Quantum computing for the very curious'
\newpage

\section{Matrix fundamentals relating to base states}

\bf{Dirac notation}
\[|0\rangle=\begin{bmatrix} 1\\0\end{bmatrix}=e_0,\:
|1\rangle=\begin{bmatrix} 0\\1\end{bmatrix}=e_1,\:
\langle0|=\begin{bmatrix} 1&0\end{bmatrix},\:\langle1|=\begin{bmatrix} 0&1\end{bmatrix}\]
\[|\psi\rangle^\dagger=\langle\psi|\]
\\~\\
\bf{Matrix addition}
\[\frac{1+i}{2}|0\rangle + \frac{i}{\sqrt{2}}|1\rangle = \begin{bmatrix} \frac{1+i}{2}\\\frac{i}{\sqrt{2}} \end{bmatrix}\]
\\~\\
\bf{Unitarity}
\[U^\dagger U=I,\quad U^\dagger=(U^T)^*,\quad
\begin{bmatrix}a & b\\c & d\end{bmatrix}^\dagger=\begin{bmatrix}a^* & c^*\\d^* & d^*\end{bmatrix},\:I=\begin{bmatrix}1&0\\0&1\end{bmatrix}\]

\section{Gates}
\subsection{Pauli gates: X(/NOT), Y and Z}
\[ NOT|0\rangle = |1\rangle \quad NOT|1\rangle = |0\rangle,\quad NOT (\alpha|0\rangle+\beta|1\rangle)=\alpha|0\rangle+\beta|1\rangle,\quad\]
\[X =\begin{bmatrix}0 & 1\\1 & 0\end{bmatrix},\:
\begin{bmatrix}0 & 1\\1 & 0\end{bmatrix}\begin{bmatrix} 1&0\end{bmatrix}=\begin{bmatrix} 0&1\end{bmatrix},\:
\begin{bmatrix}0 & 1\\1 & 0\end{bmatrix}\begin{bmatrix} 0&1\end{bmatrix}=\begin{bmatrix} 1&0\end{bmatrix}\]
\[XX|\psi\rangle=\psi\]
\[Y=\begin{bmatrix}0&-i\\i&0\end{bmatrix},\:Y|0\rangle = i|1\rangle,\:Y|1\rangle =-i|0\rangle\]
\[Z=\begin{bmatrix}1&0\\0&1\end{bmatrix},Z|0\rangle = |0\rangle,Z|1\rangle =-|1\rangle\]
\\~\\
\subsection{Hadamard Gate}
\[H|0\rangle = \frac{|0\rangle +|1\rangle}{\sqrt 2}=\begin{bmatrix}\frac{1}{\sqrt 2}\\0\end{bmatrix}+
\begin{bmatrix}0\\\frac{1}{\sqrt 2}\end{bmatrix}=
\begin{bmatrix}\frac{1}{\sqrt 2}\\\frac{1}{\sqrt 2}\end{bmatrix}\]
\[H|1\rangle = \frac{|0\rangle -|1\rangle}{\sqrt 2}=
\begin{bmatrix}\frac{1}{\sqrt 2}\\0\end{bmatrix}-
\begin{bmatrix}0\\\frac{1}{\sqrt 2}\end{bmatrix}=
\begin{bmatrix}\frac{1}{\sqrt 2}\\-\frac{1}{\sqrt 2}\end{bmatrix}
\]


\[H(\alpha|0\rangle+\beta|1\rangle)=\alpha\left(\frac{|0\rangle+|1\rangle}{\sqrt 2}\right)+\beta\left(\frac{|0\rangle-|1\rangle}{\sqrt 2}\right)
=\frac{\alpha+\beta}{\sqrt 2}|0\rangle+\frac{\alpha-\beta}{\sqrt 2}|1\rangle\]
\[H=\frac{1}{\sqrt 2}\begin{bmatrix}1&1\\1&-1\end{bmatrix},\:H=H^\dagger,\:HH=HH^\dagger=I\]
\[\Rightarrow HH|0\rangle=|0\rangle,\:HH|1\rangle=|1\rangle,\:HH|\psi\rangle=|\psi\rangle\] 
\vspace{0.5em}

Applying Hadamard to a qubit in Hadamard state, gives back the basis state Hadamard was originally applied to

\Qcircuit @C=1em @R=2em {&& \frac{\ket{0} + \ket{1}}{\sqrt 2} &&& \gate{H} & \measureD{\mbox{M}} & = & \ket{0}} \vspace{1em}
\Qcircuit @C=1em @R=2em {&& \frac{\ket{0} - \ket{1}}{\sqrt 2} &&& \gate{H} & \measureD{\mbox{M}} & = & \ket{1}}
\vspace{1em}

Two-qubit state is given by the tensor product, i.e. for below two qubit Hadamard system

$
|H|0\rangle |H|1\rangle\rangle =
H|0\rangle \otimes H|1\rangle =
\begin{bmatrix}\frac{1}{\sqrt 2}\\\frac{1}{\sqrt 2}\end{bmatrix} \otimes
\begin{bmatrix}\frac{1}{\sqrt 2}\\-\frac{1}{\sqrt 2}\end{bmatrix} =
\begin{bmatrix}\frac{1}{\sqrt 2}*\frac{1}{\sqrt 2}\\
\frac{1}{\sqrt 2}*-\frac{1}{\sqrt 2}\\
\frac{1}{\sqrt 2}*\frac{1}{\sqrt 2}\\
\frac{1}{\sqrt 2}*-\frac{1}{\sqrt 2}\end{bmatrix}=
\begin{bmatrix}\frac{1}{2}\\
-\frac{1}{2}\\
\frac{1}{2}\\
-\frac{1}{2}\end{bmatrix}
$


\subsection{CNOT Gate}
\vspace{1em}
\begin{Large}

\Qcircuit @C=1em @R=1em {
&& \lstick{\ket{1}} & \ctrl{1} & \rstick{\ket{1}} \qw &&&&& \lstick{\ket{1}} & \ctrl{1} & \rstick{\ket{1}} \qw\\
&& \lstick{\ket{1}} & \targ & \rstick{\ket{0}} \qw &&&&& \lstick{\ket{0}} & \targ & \rstick{\ket{1}} \qw
\\~\\
}
\vspace{1em}

\Qcircuit @C=1em @R=1em {
&& \lstick{\ket{0}} & \ctrl{1} & \rstick{\ket{0}} \qw &&&&& \lstick{\ket{0}} & \ctrl{1} & \rstick{\ket{0}} \qw\\
&& \lstick{\ket{1}} & \targ & \rstick{\ket{1}} \qw &&&&& \lstick{\ket{0}} & \targ & \rstick{\ket{0}} \qw
\\~\\
}

\bf{In general superposition, doesn't seem to modify control bit\\}
\Qcircuit @C=1em @R=1em {&&&&&&&&&&&&& \alpha\ket{00}+\beta\ket{01}+\gamma\ket{10}+\delta\ket{11}
\rightarrow \alpha\ket{00}+\beta\ket{01}+\gamma\ket{11}+\delta\ket{10}}
\vspace{1em}

\bf{But in this specific Hadamard case, seemingly does it\\}
\Qcircuit @C=1em @R=1em {
&&& \lstick{\frac{\ket{0} + \ket{1}}{\sqrt 2}} & \ctrl{1} & \qw & \rstick{\frac{\ket{0} - \ket{1}}{\sqrt 2}} \qw \\
&&& \lstick{\frac{\ket{0} - \ket{1}}{\sqrt 2}} & \targ & \qw  & \rstick{\frac{\ket{0} - \ket{1}}{\sqrt 2}} \qw
}
\vspace{1em}

\[\frac{1}{\sqrt 2}\begin{bmatrix}0\\1\end{bmatrix}\]
\end{Large}

\vspace{1em}
\subsection{Rotation}
\[R_y(\theta) =\begin{bmatrix}\cos(\theta/2) & -\sin(\theta/2)\\ \sin(\theta/2) & \cos(\theta/2)\end{bmatrix}\]
\end{document}
